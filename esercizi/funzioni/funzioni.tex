\documentclass[12pt]{article}

\usepackage{sectsty}
\usepackage{graphicx}

% Margins
\topmargin=-0.45in
\evensidemargin=0in
\oddsidemargin=0in
\textwidth=6.5in
\textheight=9.0in
\headsep=0.25in

\title{ Funzioni }
\author{ Andrea Canale }
\date{\today}

\begin{document}
\maketitle	

%--Paper--
\section{Es 19 - Pag 96}


$  f(x) = 2x^2 + 3x - 1 $

$ f(2x) = 8x^2 + 6x - 1 $

$ 2f(x) = 4x^2 + 6x - 2 $

$ f(x + 1) = 2x^2 + 7x + 4 $

$ f(x) + 1 = 2x^2 + 3x $

\section{Es 20 - Pag 96}


$  f(x) = \frac{x-2}{x+1} $

$ f(x+2) = \frac{x}{x+3} $

$ f(x) + 2 = \frac{3x}{x+1} $

$ f(2x) = \frac{2x -2}{2x + 1} $

$ 2f(x) = \frac{2x-4}{2x + 1} $

% Sbagliato, da correggere %
\section{Es 29 - Pag 97}

$ f(x) = \frac{x-1}{x+1}$
$ g(x) = 2x $

$ 2f(x+1) >= 2g(x-1)-3 $

$ \frac{2x}{x+1} >= 4x -4 $

$ \frac{2x}{x+1} - 4x + 4 >= 0$

$ \frac{-4x^2+2x+4}{x+1} >= 0 $

Risoluzione

\,

$ -4x^2 + 2x + 4 >= 0 $

\,

$ \frac{-2 \pm \sqrt{2^2 - 4 \cdot (-4) \cdot 4}}{-8} = \frac{-2 \pm \sqrt{4 + 64}}{-8} $

\,

$ \frac{-2 + \sqrt{68}}{-8} $

$ \frac{-2 - \sqrt{68}}{-8} $

\,


$ x + 1 >= 0 $

$ x >= -1 $

\pagebreak

\end{document}