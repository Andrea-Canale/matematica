\documentclass[11pt]{article}

\usepackage{sectsty}
\usepackage{graphicx}

% Margins
\topmargin=-0.45in
\evensidemargin=0in
\oddsidemargin=0in
\textwidth=6.5in
\textheight=9.0in
\headsep=0.25in

\title{ Es 1}
\author{ Author }
\date{\today}

\begin{document}
\maketitle	

%--Paper--
\section{Es 257 - Pag 705}

$\log{x^2} + \frac{1}{\log{x}} = 3$

$\log{x^2} + \frac{1}{\log{x}} - \log{1000} = 0$

\section{Es 376 - Pag 712}

$\log{(x + 1)} + \log{(3 - x)} > 2\log{2}$

Condizioni di esistenza:
$ 
\left \{ \begin{array}{rl}
    x + 1 > 0 \\
    3 - x > 0
    \end{array}
    \right.
$

$ 
\left \{ \begin{array}{rl}
    x > -1 \\
    x < 3
    \end{array}
    \right.
$

\,

C.E = $ x > -1 \vee x < 3 $

\,

$\log{(-x^2 + 2x + 3)} > 2\log{2} $

$ -x^2 + 2x - 1 > 0 $

\,

Risoluzione:

\,

$\frac{-2 \pm \sqrt{2^2 - 4 \cdot (-1 \cdot -1)}}{-2} $

$\frac{-2 \pm \sqrt{4 - 4}}{-2} $

$\frac{-2 }{-2} = 1$


% Nuovo esercizo %

\section{ Es 377 - Pag 712 }

$\log{( 5 - x )} + \log{\frac{x}{2}} \geq log{(x - 2)}$

Condizioni di esistenza:
$ 
\left \{ \begin{array}{rl}
    5 - x > 0 \\
    \frac{x}{2} > 0 \\
    x - 2 > 0
    \end{array}
    \right.
$

$ 
\left \{ \begin{array}{rl}
    x < 5 \\
    x > 0 \\
    x > 2
    \end{array}
    \right.
$


\,

$ \log{\frac{-x^2 + 5x}{2}} \geq \log(x-2) $

$ \frac{-x^2 + 5x}{2} \geq x-2 $

$ -x^2 + 3x + 4 > 0 $

$ \frac{-3 \pm \sqrt{3 ^ 2 - 4(-1 \cdot 4)}}{-2} = 0$

$ \frac{-3 \pm \sqrt{25}}{-2} = 0 $

$ \frac{-3 \pm 5}{-2} = 0 $

$ \frac{-3 + 5}{-2} = -1 $

$ \frac{-3 - 5}{-2} = 4 $

\,

Risultato: $ 4 > x > 2 $

\section{ Es 389 - Pag 712 }

$ \ln^2{x} + \ln{x} > 0 $

Condizioni di esistenza:
$ 
\left \{ \begin{array}{rl}
    x > 0 \\
    \end{array}
    \right.
$

$ \ln{x} = t $

$ t^2 + t > 0 $

$ \frac{-1 \pm \sqrt{1^2 - 4 \cdot (1 \cdot 0)}}{2} $

\,

$ \frac{-1 + 1}{2} = \frac{0}{2} = 0 $

\,

$ \frac{-1 - 1}{2} = \frac{-2}{2} = 1 $

$ t < 0 \vee t > 1 $

$ \ln{x} < 0 \vee \ln{x} > 1 $

$ \ln{x} < 0 || x = e^0 $

$ \ln{x} > 1 || x = e^1 $

Risultato: $ x < 1 \vee x > e $

\section{ Es 391 - Pag 712 }

$ \ln^2{x} + \ln{x} < 2 $

$ \ln{x} = t $

$ t^2 + t > 2 $

$ t^2 + t -2 > 0 $

$ \frac{-1 \pm \sqrt{1^2 - 4 \cdot (1 \cdot - 2)}}{2} $

$ \frac{-1 \pm \sqrt{1 + 8}}{2} $

\,

$ \frac{-1 + 3}{2} = 2 $

\,

$ \frac{-1 - 3}{2} = -2 $

$ t < -2 \vee t > 2 $

\,

\pagebreak

Ritorno a $ \ln{x} $

\,

$ \ln{x} < -2 \vee \ln{x} > 2 $

\, 

1. $ \ln{x} < -2 $

$ \ln{x} < \ln{e^{-2}} $

$ x < e^{-2} $

\, 

2. $ \ln{x} > \ln{e^2} $

$ x > e^2 $

Risultato: $ x < e^{-2} \vee x > e^2 $
\pagebreak

\end{document}